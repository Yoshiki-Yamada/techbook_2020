\documentclass[autodetect-engine,dvipdfmx-if-dvi,ja=standard,b5paper,10.5pt,twoside,openany,layout=v2]{bxjsbook}

\newcommand{\stypath}{./sty}
\newcommand{\articlepath}{./articles}
\newcommand{\assetspath}{./assets}

\newcommand{\lrfasset}{\assetspath/lrf141asset}
\newcommand{\chikuwaitasset}{\assetspath/chikuwa_ITasset/gray}
\newcommand{\strvertasset}{\assetspath/strvertasset}

\usepackage{\stypath/localst17}
\usepackage{\stypath/mymintedsetting}

\usepackage{lipsum}
\usepackage{layout}

%keisuke495500
\usepackage{caption}

%Takuzoo3868
\usepackage{dirtree}

%Jumpaku
\usepackage{pxrubrica}
\usepackage{hyperref}
\usepackage{pxjahyper}
\usepackage{comment}
\usepackage{verbatim}

%materialofmouse
\usepackage{siunitx}

\usepackage{amsmath}
\usepackage{graphicx}

\title{情報ボーイズの寄稿ノート}
\author{うっひょい \and ちくうぇいと \and あわあわ \and けんつ \and さわだ \and Jumpaku \and あるねこ}

\date{}

\begin{document}
\frontmatter
\maketitle
\begin{myintroduce}{\chikuwaitasset/icon.jpg}{ちくうぇいと @chikuwa\_IT}
  RubyでイケイケWEBエンジニアになるつもりだったのに, 気がついたらmrubyをOSの下に組み込んだりして完全にシステムプログラミング沼に落ちてしまいました.\\
  どうしてこうなった.
\end{myintroduce}
\begin{myintroduce}{\lrfasset/icon_gray.jpg}{けんつ @lrf141}
  カーネルからWebまで色々やってます.\\
  インフラとScalaとJavaが大好物です.
\end{myintroduce}


\chapter{はじめに}
\addtolength{\oddsidemargin}{10pt}
\addtolength{\evensidemargin}{-10pt}
%\addtolength{\textwidth}{-14pt}
LOCALがくせーぶの紹介とかなんか.部長!お願いしますよ! by さわだ


\tableofcontents
\mainmatter
%\addtolength{\oddsidemargin}{-20pt}
%\addtolength{\evensidemargin}{18pt}

\chapterauthor{chikuwait(ちくうぇいと)}
\chapter{超入門 仮想化技術}
\section{はじめに}
近年, コンテナ型仮想化の普及もあり仮想化技術という存在を様々な場面でよく聞くようになりました. しかし, コンテナ型仮想化技術以外の他の仮想化技術というのは中々個人で扱うような技術・ソフトウェアではないこともあり, あまり知られていないことが多かったりします. また, コンテナ型仮想化技術も便利で環境構築が魔法のようなすごい存在といった漠然とした解釈の人もきっと多いはずです. そこでこの章では, 各種仮想化技術の仕組みや特徴について触れながらふわっと仮想化技術について紹介します.

\section{仮想化技術とは}
仮想化技術にはいくつか種類がありますが, この章では計算機資源を抽象化してOSなどに見せるプラットフォーム仮想化のことを指します. 仮想化することで複数の計算機資源を単一に見せたり, 単一の計算機資源を複数に見せることができます. そしてプラットフォーム仮想化を支えるための技術としてシステム仮想機械(VM, Virtual Machine)と呼ばれる計算機資源をエミュレートするソフトウェアが存在します. 仮想機械の実装は仮想マシンモニタ(VMM, Virtual Machine Monitor)を始めとしていくつか存在します. 例えば仮想専用サーバ(VPS, Virtual Private Server)のようなサービスでは図2.1にように, 物理サーバ上に仮想化OSによって複数の仮想サーバに分割してユーザに各仮想サーバを提供しています.
\begin{figure}[htbp]
    \centering
    \includegraphics[width=50mm]{./assets/chikuwa_ITasset/vps.png}
    \caption{hogehoge}
    \label{fig:one}
\end{figure}

\section{VMMについて}
仮想化を実現するVMMにはハイパーバイザ型(Type1),ホスト型(Type2)の2つに分類されます.

\subsection{ハイパーバイザー型(Type1)}
ハイパーバイザ型(Type1)は, ハードウェアの上で直接動作します. この方式ではホストOSと呼ばれるような土台になるOSが存在しないため,仮想マシンによる遅延や速度低下を防ぐことができます. そしてベアメタルハイパーバイザでは実装手法でもモノリシックカーネル型とマイクロカーネル型の2つに分類することができるほか, 仮想化のアプローチで完全仮想化と準仮想化に分類することができます.
\subsubsection{モノリシックカーネル型}
主にVMware ESX/ESXiなどで採用されている方式で, モノリシックという英語で「1枚岩」という意味の通り、VMMの中にデバイスドライバが含まれています. VMMがストレージをはじめ,ネットワークや入力デバイスといったハードウェアへのアクセスをすべてを処理します. この方法の利点はVMMとデバイスドライバが密接に連携するため,オーバーヘッドが少なく効率的です. しかしながら, VMMの中にデバイスドライバが存在しているため,ハイパーバイザ層でデバイスドライバを用意する必要があります. そのため, ハードウェアのサポートがマイクロカーネル型と比較して少なく, 使用するハードウェアに制限がかかってしまう場合があります. また, デバイスドライバをVMMに直接組み込むため, バグや脆弱性はVMM全体に広がってしまいます.
\begin{figure}[htbp]
    \centering
    \includegraphics[width=50mm]{./assets/chikuwa_ITasset/monolithic.png}
    \caption{モノリシックカーネル型ベアメタルハイパーバイザ}
    \label{fig:one}
\end{figure}
\begin{figure}[htbp]
    \centering
    \includegraphics[width=50mm]{./assets/chikuwa_ITasset/monolithic.png}
    \caption{モノリシックカーネル型のハードウェアアクセス}
    \label{fig:two}
\end{figure}
\subsubsection{マイクロカーネル型}
主にXenやHyper-Vで採用されている方式で, ハイパーバイザを管理する仮想マシンと管理OSを用意します. この管理OSはLinuxはWindows Serverなど汎用OSを使用します. また, 管理OSはXenではドメイン0, Hyper-Vでは親パーティションと呼ばれています. この方式では, デバイスドライバはVMMではなく, VMM上の仮想マシンとして動作している管理OSのデバイスドライバを使用します. 仮想マシンからハードウェアにアクセスする時はゲストOSから仮想デバイスのインターフェースを経由してVMMから管理OSに渡されます. そして管理OSのデバイスドライバからハードウェアにアクセスします.この方法は, 汎用OSのデバイスドライバを使用することで, モノリシックカーネル型に比べてハードウェアのサポートが多く, ハードウェアの対応が柔軟であるという利点があります. 例えば, Hyper-VならWindows用のデバイスドライバを使用することができます. しかしながら, ハードウェアにアクセスする際にVMMから管理OSを経由するため, モノリシックカーネル型よりも性能が低下してしまいがちであり, 管理用の汎用OSがクラッシュした場合全てのVMがクラッシュしてしまうといった欠点が存在します.
\begin{figure}[htbp]
    \centering
    \includegraphics[width=50mm]{./assets/chikuwa_ITasset/monolithic.png}
    \caption{マイクロカーネル型ベアメタルハイパーバイザ}
    \label{fig:one}
\end{figure}
\begin{figure}[htbp]
    \centering
    \includegraphics[width=50mm]{./assets/chikuwa_ITasset/monolithic.png}
    \caption{マイクロカーネル型のハードウェアアクセス}
    \label{fig:two}
\end{figure}

\subsubsection{完全仮想化}
完全仮想化方式のVMMでは, ハードウェアの挙動をすべてエミュレートします. そのため, 何も変更も加えていないそのままのホストOSを動かすことができます. 1960年代にIBMが「トラップアンドエミュレート」とよばれる方法で完全仮想化を実装しようとしました. この方法ではゲストOSが特権がない状態(Ring3)で実行させ, 特権(Ring0)が必要な命令を実行しようとすると失敗します. その際にVMMがその失敗をトラップして原因を確認してからその命令をエミュレートすることによってゲストOSの期待する結果を返すことができ, ゲストOSにRing0以外で実行されていることを気づかせないようにすることができます. しかしながら, この手法は古典的ですべてのアーキテクチャに適用できるわけではありませんでした. 特にx86プロセッサの場合, ユーザ権限で実行できるセンシティブ命令と呼ばれる計算機資源の構成などの依存している命令が存在しているため, 実装を難しくさせていました. そこで, 「バイナリトランスレーション」と呼ばれる新しい手法が使われるようになりました. この手法では, センシティブな命令以外の命令は直接CPUで実行し, センシティブな命令はVMMで実行前に動的に他の命令に置き換えられます.
\begin{figure}[htbp]
    \centering
    \includegraphics[width=50mm]{./assets/chikuwa_ITasset/monolithic.png}
    \caption{バイナリトランスレーション}
    \label{fig:one}
\end{figure}
\subsubsection{準仮想化}
\subsection{ホスト型(Type2)}


\chapterauthor{けんつ}
\chapter{入門Linux Kernel}
\section{はじめに}

最近の開発、主にその開発環境やインフラレイヤーにおいて無くてはならないものとしてDockerが挙げられる。そして、Dockerを使ってWebサービスのインフラ等を構築する場合よく利用するのはdocker-composeだろう。
この章ではDockerの基本的な部分から紹介し、docker-composeの裏側を解説したあとにDocker Remote APIを使って自分でコンテナを制御するということについて紹介する。


\section{Dockerとは}

Dockerそのものに関しては公式ドキュメントよりもさくらインターネットさんの「さくらのナレッジ Docker入門(第一回)~Dockerとは何か、何が良いのか~」で簡単に紹介されていたのでそちらを引用する。
Dockerとはその記事で次のように紹介されている。\cite{sakura}

\begin{quote}
Dockerは、インフラ関係やDevOps界隈で注目されている技術の一つで、Docker社が開発している、コンテナ型の仮想環境を作成、配布、実行するためのプラットフォームです。
(https://www.docker.com/what-docker)

Dockerは、Linuxのコンテナ技術を使ったもので、よく仮想マシンと比較されます。VirtualBoxなどの仮想マシンでは、ホストマシン上でハイパーバイザを利用しゲストOSを動かし、その上でミドルウェアなどを動かします。それに対し、コンテナはホストマシンのカーネルを利用し、プロセスやユーザなどを隔離することで、あたかも別のマシンが動いているかのように動かすことができます。そのため、軽量で高速に起動、停止などが可能です。
\end{quote}

これが全てとなる。Dockerコンテナを利用することで特にWebサービス開発の開発環境において共通のインフラをチーム全員で共有することができるなどの様々なメリットがある。


\subsection{Dockerのアーキテクチャ}
Dockerとはコンテナの色々に関わるプラットフォームを指しているが実際にコンテナを利用する場合にはdockerコマンドを通して使う場合が多い。
dockerコマンドは何をしているかというと、docker remote apiを呼び出している。このdocker remote apiなるものは、Dockerのアーキテクチャに深く関係している。


Dockerのアーキテクチャはクライアント・サーバアーキテクチャとなっておりdockerコマンドはただのCLIツールである。
そのただのCLIツールが何をしているかというとdocker daemonが提供しているREST APIライクなDocker Remote APIをコールしているに過ぎない。


Docker Remote APIはデフォルトでunixドメインソケットを用いることで利用できるようになっている。自身で設定することでtcpによるリクエストを許可することも出来る。これ移行の章ではそれを有効にしている前提で話を進めていく。

\section{Docker Composeとは}

Docker-Composeとは、前述のDockerコンテナを複数利用する場合にそれらをまとめて管理するためのツールである。
Dockerコマンドだけでは何が問題かというと、例えばweb開発においてインフラをDockerコンテナで用意するならばNginxのコンテナにMySQLのコンテナ、PHPやRailsが動作するコンテナなど複数のコンテナが必要になってくる。
これら全てをdocker runで呼び出すとシェルなどにまとめておく必要がある。そして利便性に欠ける。では、サービスに必要なコンテナ群を一箇所でまとめて管理してしまいたい。そういう場合にdocker-composeを利用するのは非常に有効な手段となる。


そして、Docker ComposeはDocker Remote APIを通してdocker-compose.ymlに記載されているコンテナ群を管理している。
この点がこれ移行の説明で非常に重要になっていく。

\section{docker-composeを自作する}

ここまで紹介した情報を元にこの章の本題である、Dockerコンテナの制御を行う。制御といっても何をするかというとDocker Remote APIを使って超絶最高にシンプルなdocker-composeを作ってみる。Golangで。書いたことあまりないけど。


それではまず要件を以下に定義する。
\begin{itemize}
    \item ymlを解析して作成するコンテナやボリューム、ポートを設定できる
    \item コンテナを走らせる。
    \item コンテナを落とす。
    \item 起動しているコンテナのステータスを取る
\end{itemize}

\subsection{コマンドライン引数をパースする}

まずは、cliツールを作る上でコマンドライン引数をパースする必要がある。そこで import flag を使って簡易的にパースしていく。今回のcliツールではup,down,statusまでできれば十分なのでそれだけ定義する。

\begin{minted}[frame=lines,framesep=2mm,baselinestretch=1.2,fontsize=\footnotesize,linenos,breaklines]{go}
package main

import (
    "flag"
    "fmt"
)

func main() {
    flag.Parse()

    switch flag.Arg(0) {

    case "up":
        fmt.Println("up")
    case "down":
        fmt.Println("down")
    case "status":
        fmt.Println("status")
    default:
        fmt.Println("unknown")
    }
}
\end{minted}

\subsection{ymlをパースする}

次にそれなりに重要な機能であるymlのパースを行う。ただパーサーから書いている時間はなかったのでここでは gopkg.in/yaml.v2 というパッケージを使用する。

\begin{minted}[frame=lines,framesep=2mm,baselinestretch=1.2,fontsize=\footnotesize,linenos,breaklines]{go}
package main

import (
    "flag"
    "fmt"
    yaml "gopkg.in/yaml.v2"
    "io/ioutil"
)

type Services struct {
    Services []Docker `yaml:"services"`
}

type Docker struct {
    Name string     `yaml:"name"`
    Image string    `yaml:"image"`
    Command string  `yaml:"command"`
}

func main() {

    buf, err := ioutil.ReadFile("./lite-compose.yml")
    if err != nil {
        panic(err)
    }

    var parsed Services
    err = yaml.Unmarshal(buf, &parsed)
    if err != nil {
        panic(err)
    }

    fmt.Println(parsed)

    flag.Parse()
    switch flag.Arg(0) {

    case "up":
        fmt.Println("up")
    case "down":
        fmt.Println("down")
    case "status":
        fmt.Println("status")
    default:
        fmt.Println("unknown")
    }
}
\end{minted}

コード自体はこれでいいのだがひとつ問題がある。docker-composeのようなパターンのymlを今回使ったライブラリではパースできないことだ。
しかし、パーサーから探すのはめんどくさいので仕方なくymlの形式を以下のようにして lite-compose.yml とする。

\begin{minted}[frame=lines,framesep=2mm,baselinestretch=1.2,fontsize=\footnotesize,linenos,breaklines]{yaml}

services:
-
  name: ubuntu
  image: ubuntu
  command: 'echo "hello, world"'

\end{minted}

これを実行すると構造体として表示される。ただこれだけでは色々と不十分なのでエラーハンドリングを追加してファイルを分割してみた。

\begin{minted}[frame=lines,framesep=2mm,baselinestretch=1.2,fontsize=\footnotesize,linenos,breaklines]{go}
package main

import (
    "flag"
    "fmt"
    "gopkg.in/yaml.v2"
    "io/ioutil"
)

func main() {

    buf, err := ioutil.ReadFile("./lite-compose.yml")
    if err != nil {
        panic(err)
    }

    var parsed Services
    err = yaml.Unmarshal(buf, &parsed)
    if err != nil {
        panic(err)
    }

    if err := requireNotExist(); !parsed.validate() {
        panic(err)
    }

    flag.Parse()
    switch flag.Arg(0) {

    case "up":
        fmt.Println("up")
    case "down":
        fmt.Println("down")
    case "status":
        fmt.Println("status")
    default:
        fmt.Println("unknown")
    }
}
\end{minted}

\begin{minted}[frame=lines,framesep=2mm,baselinestretch=1.2,fontsize=\footnotesize,linenos,breaklines]{go}
package main

import "errors"

type Services struct {
    Services []Docker `yaml:"services"`
}

type Docker struct {
    Name string     `yaml:"name"`
    Image string    `yaml:"image"`
    Command string  `yaml:"command"`
}

func (services *Services) validate() bool {
    for _, docker := range services.Services {
        if docker.Image == "" || docker.Name == "" {
            return false
        }
    }
    return true
}


func requireNotExist() error {
    return errors.New("The syntax of yaml is incorrect")
}
\end{minted}

\section{コンテナを制御する}

ここまでやってきたら後はdocker remote apiを使ってdockerコンテナを立ち上げ落としてステータスを取れるようにする。
docker remote api自体はunixドメインソケットかtcpまたはhttpでリクエストを飛ばすことで利用することができる。ここではそれらを順を追って説明していく。
本来はそれらリクエスト部分を自作するつもりであったが今回は golang でdocker remote apiを利用できるライブラリのgithub.com/docker/dockerを使用する。


実際にdocker remote apiを使ったコンテナ制御は以下のようになる。
\begin{minted}[frame=lines,framesep=2mm,baselinestretch=1.2,fontsize=\footnotesize,linenos,breaklines]{go}
package main

import (
    "context"
    "fmt"
    "github.com/docker/docker/api/types"
    "github.com/docker/docker/api/types/container"
    "github.com/docker/docker/api/types/network"
    "github.com/docker/docker/client"
    "os"
    "strings"
    "time"
)

func createContainer(services Services) {
    ctx := context.Background()
    cli, err := client.NewClientWithOpts(client.WithVersion("1.39"))
    if err != nil {
        panic(err)
    }

    p, _ := os.Getwd()
    split := strings.Split(p, "/")
    current := split[len(split)-1]

    for _, service := range services.Services {

        hostConfig := &container.HostConfig{}
        containerConfig := &container.Config{
            Image: service.Name,
            Tty: true,
            Cmd: strings.Split(service.Command, " "),
        }
        networkConfig := &network.NetworkingConfig{}

        resp, err := cli.ContainerCreate(ctx, containerConfig, hostConfig, networkConfig, current+"_"+service.Name)
        if err != nil {
            fmt.Println(err)
        } else {
            err := cli.ContainerStart(ctx, resp.ID, types.ContainerStartOptions{})
            if err != nil {
                fmt.Println(err)
            } else {
                fmt.Println(resp)
            }

        }
    }
}

func killContainer(services Services) {
    ctx := context.Background()
    cli, err := client.NewClientWithOpts(client.WithVersion("1.39"))
    if err != nil {
        panic(err)
    }

    p, _ := os.Getwd()
    split := strings.Split(p, "/")
    current := split[len(split)-1]

    timeout := 5 * time.Second

    for _, service := range services.Services {
        resp := cli.ContainerStop(ctx, current+"_"+service.Name, &timeout)
        fmt.Println(resp)
    }
}

func getDockerImageList() {

    ctx := context.Background()
    cli, err := client.NewClientWithOpts(client.WithVersion("1.39"))
    if err != nil {
        panic(err)
    }

   list, err := cli.ImageList(ctx, types.ImageListOptions{})
   if err != nil {
        panic(err)
   }

   for _, image := range list {
       fmt.Printf("RepoTags: %s, Labels: %s, Container: %d, Size: %d\n", image.RepoTags, image.Labels, image.Containers, image.Size)
   }

}


func getDockerPs() {
    ctx := context.Background()
    cli, err := client.NewClientWithOpts(client.WithVersion("1.39"))
    if err != nil {
        panic(err)
    }

    containers, err := cli.ContainerList(ctx, types.ContainerListOptions{
        All: true,
    })

    if err != nil {
        panic(err)
    }

    for _, container := range containers {
        fmt.Printf("ID: %s, Name: %sm Image: %s, Command: %s\n", container.ID, container.Names, container.Image, container.Command)
    }
}
\end{minted}

ここでは、ライブラリを使用してHostConfigなどを設定し、コンテナを作成した上で起動させるという一連の流れを行っている。
さらにportsやvolumesなど今回は導入しなかったが通常docker-compose.ymlで設定するようなパラメータ群を設定することもできるので興味があれば実際に試して欲しい。

\section{おわりに}

ここで紹介したように、dockerとはプラットフォームであり普段使うようなdockerコマンドは実際にはdocker remote apiを叩いているに過ぎない。また、docker-composeはそれらをまとめて管理しているがこれもまたdocker remote apiを叩いているだけである。
そのため、これらを応用すれば競技プログラミングやオンライン実行環境で実際にコードを動かす部分をユーザのアクションに併せて作成することができる。
是非そのようなサービスを開発する際にここに書いてある知識を応用して貰えると締め切りの前日からこのようなものを作り始めた甲斐があるというものである。


\begin{thebibliography}{10}
    \bibitem{sakura} Docker入門(第一回)~Dockerとは何か、何が良いのか~ | さくらのナレッジ https://knowledge.sakura.ad.jp/13265/
    \bibitem{unixdomain} tcp-hist.ps http://osnet.cs.binghamton.edu/publications/TR-20070820.pdf
\end{thebibliography}



\chapterauthor{すとんりばー}
\chapter{後で変える}
\section{はじめに}


\newpage
\myimpression[%
name=LOCAL Students\\情報ボーイズの寄稿ノート, %
author=うっひょい, \and %
ちくうぇいと, \and %
あわあわ, \\ \and %
けんつ, \and %
さわだ, \and %
Jumpaku, \and %
あるねこ, %
date=2018年4月22日, %
publisher=LOCAL学生部, %
print=有限会社ねこのしっぽ %
]%
\end{document}
