\section{はじめに}
自宅サーバー,憧れますよね.そんな自宅サーバーをお手頃な価格で自作してしまいましょう.せっかくなのでモーターとタイヤをつけてしまいましょう.そうです,自走式WEBサーバーです.この章ではWiFiを扱うことのできるマイコン ESP-WROOM-02を用いた自走式WEBサーバーを作成します.
\footnote{本章に記載されたプログラム,回路図,その他工作物を参考にして製作した場合に生じた事故等について一切の責任を負いかねます.}
\subsection{ESP-WROOM-02とは}
ESP-WROOM-02は,上海の企業Espressif Systems のESP8266EXチップを搭載したWiFiモジュールです.いわゆる「技適」を取得しているため,日本国内で法的に安心して使えます.ardiono言語で開発できることに加え,おおよそ600円で購入できる手軽さが魅力です.
\footnote{ESP-WROOM-02の開発環境の構築方法については, ESP8266 Arduino Coreの公式ドキュメント(\url{https://arduino-esp8266.readthedocs.io/en/latest/})や僕のブログ(\url{https://haibara-works.hatenablog.com})の『ESP-WROOM-02を導入する』を参照してください.(ページ数の都合から割愛します)}

\subsection{ハードウェアの準備}
表\ref{buhin}に今回作る自走式WEBサーバーに必要な部品を示します.
\begin{table}[htb]
\centering
\caption{使うもの}
\begin{tabular}{|l|l|} \hline
部品 & 用途 \\ \hline \hline
ESP-WROOM-02 & ESP本体  \\ \hline
FT-232RQ & USBシリアル変換モジュール  \\ \hline
抵抗(10 K$\Omega$) & プルアップ・プルダウン用  \\ \hline
抵抗(470 K$\Omega$) & LEDの保護抵抗 \\ \hline
LED & パイロットランプ  \\ \hline
TA48033S & 電源レギュレータ(3.3 V) \\ \hline
コンデンサ(0.33 $\mu$F) & 電源用パスコン  \\ \hline
コンデンサ(33 $\mu$F) & 電源用パスコン  \\ \hline
TA7291P & モータードライバ \\ \hline
ダブルギヤボックス(左右独立4速タイプ) & タミヤのギヤボックス  \\ \hline
ブレッドボード & テスト用 \\ \hline
ジャンパ線 & テスト用 \\ \hline
\end{tabular}
\label{buhin}
\end{table}

\section{WEBサーバーとしての動作}

    \subsection{SPIFFSの利用}


\begin{minted}[frame=lines,framesep=2mm,baselinestretch=1.2,fontsize=\footnotesize,linenos,breaklines]{c}
#include <ESP8266WiFi.h>
#include <WiFiClient.h>
#include <ESP8266mDNS.h>
#include <ESP8266WebServer.h>
#include <ESP8266HTTPUpdateServer.h>
#include <FS.h>
#include "config.h"

const char* host = "esp8266fs";
ESP8266WebServer Server(80);      //80番ポートを使用

/**************************************************
  WEBサーバーに関する関数群
 **************************************************/
/**  Show URI args */
void showUriArgs() {
  Serial.printf("\n---URI args---\n");
  for (int i = 0; i < Server.args(); i++) {
    Serial.printf("%s: %s \n", Server.argName(i).c_str(), Server.arg(i).c_str());
  }
}

/**  ファイルの拡張子を調べてMIMEタイプを返す関数 */
String getContentType(String filename) {
  if (Server.hasArg("download")) return "application/octet-stream";
  else if (filename.endsWith(".htm")) return "text/html";
  else if (filename.endsWith(".html")) return "text/html";
  else if (filename.endsWith(".css")) return "text/css";
  else if (filename.endsWith(".js")) return "application/javascript";
  else if (filename.endsWith(".png")) return "image/png";
  else if (filename.endsWith(".gif")) return "image/gif";
  else if (filename.endsWith(".jpg")) return "image/jpeg";
  else if (filename.endsWith(".ico")) return "image/x-icon";
  else if (filename.endsWith(".xml")) return "text/xml";
  else if (filename.endsWith(".pdf")) return "application/x-pdf";
  else if (filename.endsWith(".zip")) return "application/x-zip";
  else if (filename.endsWith(".gz")) return "application/x-gzip";
  else return "text/plain";
}

/**  指定されたパスのファイルをクライアントに送信 */
void handleSendRes(void) {
  showUriArgs();
  String path = Server.uri();

  if (path.equals("/motor.html")) {
    motor._mode = Open;
    if (Server.arg("motorMode").equals("forw")) {
      motor._mode = Forw;
    } else if (Server.arg("motorMode").equals("back")) {
      motor._mode = Back;
    } else if (Server.arg("motorMode").equals("right")) {
      motor._mode = Right;
    } else if (Server.arg("motorMode").equals("left")) {
      motor._mode = Left;
    }
    motor.val = atoi(Server.arg("motorVal").c_str());
  }

  Serial.println("");
  Serial.println("[handleSendRes]: trying to read " + path);

  if (path.endsWith("/")) path += "index.html";

  String contentType = getContentType(path);

  if (SPIFFS.exists(path)) {
    Serial.println("[handleSendRes]: sending " + path);
    File file = SPIFFS.open(path, "r");
    Server.streamFile(file, contentType);
    file.close();
    Serial.println("[handleSendRes]: sent " + path);
  } else {
    Serial.println("[handleSendRes]: 404 not found");
    Server.send (404, "text/plain", "ESP: 404 not found");
  }
}

/**  settings */
void setup() {
  motor._mode = Open;
  motor.val = 0;

  Serial.begin(74880);

  pinMode(Pilot, OUTPUT);
  digitalWrite(Pilot, HIGH);

  motorSetup(M1_l, M1_r, M2_l, M2_r);

  WiFi.mode(WIFI_STA);
  WiFi.begin(ssid, pass);
  delay(100);

  Serial.println("");

  while (WiFi.status() != WL_CONNECTED) {
    delay(500);
    digitalWrite(Pilot, !digitalRead(Pilot));
    Serial.print(".");
  }
  digitalWrite(Pilot, HIGH);

  Serial.println("");
  Serial.print("Connected to ");
  Serial.println(ssid);
  Serial.print("IP address: ");
  Serial.println(WiFi.localIP());
  MDNS.begin(host);

  SPIFFS.begin();
  {
    Dir dir = SPIFFS.openDir("/");
    while (dir.next()) {
      String fileName = dir.fileName();
      size_t fileSize = dir.fileSize();
      Serial.printf("FS File: %s, size: %s\n", fileName.c_str(), formatBytes(fileSize).c_str());
    }
    Serial.printf("\n");
  }

  //  ウェブサーバの設定
  Server.on("/list", HTTP_GET, handleFileList);
  Server.on("/edit", HTTP_GET, []() {
    if (!handleFileRead("/edit.htm")) {
      Server.send(404, "text/plain", "FileNotFound");
    }
  });
  Server.on("/edit", HTTP_PUT, handleFileCreate);
  Server.on("/edit", HTTP_DELETE, handleFileDelete);
  Server.on("/edit", HTTP_POST, []() {
    Server.send(200, "text/plain", "");
  }, handleFileUpload); 
  Server.on("/all", HTTP_GET, []() {
    String json = "{";
    json += "\"heap\":" + String(ESP.getFreeHeap());
    json += ", \"analog\":" + String(analogRead(A0));
    json += ", \"gpio\":" + String((uint32_t)(((GPI | GPO) & 0xFFFF) | ((GP16I & 0x01) << 16)));
    json += "}";
    Server.send(200, "text/json", json);
    json = String();
  });
  Server.onNotFound(handleSendRes);  ////called when handler is not assigned
  
  Server.begin();
}

/**  main loop */
void loop() {
  Server.handleClient();
  MDNS.update();

  switch (motor._mode) {
    case Forw:
      goForward(motor.val);
      break;
    case Back:
      goBack(motor.val);
      break;
    case Right:
      turnRight(motor.val);
      break;
    case Left:
      turnLeft(motor.val);
      break;
    case Open:
    default:
      stopMotor();
  }
}
\end{minted}


\section{ラジコンとしての動作}
%    \subsubsection{}

\begin{minted}[frame=lines,framesep=2mm,baselinestretch=1.2,fontsize=\footnotesize,linenos,breaklines]{c}
enum { M1_l = 4, M1_r = 5, M2_l = 12, M2_r = 13, Pilot = 16,
       Stop, Open, Forw, Back, Right, Left };

typedef struct {
  int _mode;
  int val;
} motor_T;
motor_T motor;

/**************************************************
  モーター制御に関する関数群
 **************************************************/
/**  左右2つのモーターに接続されるピンを初期化 */
void motorSetup(const int m1_l, const int m1_r, const int m2_l, const int m2_r) {
  pinMode(m1_l, OUTPUT);
  pinMode(m1_r, OUTPUT);
  pinMode(m2_l, OUTPUT);
  pinMode(m2_r, OUTPUT);

  digitalWrite(m1_l, LOW);
  digitalWrite(m1_r, LOW);
  digitalWrite(m2_l, LOW);
  digitalWrite(m2_r, LOW);

  analogWriteFreq(5);
}

/**  モーターをPWM制御する関数 */
void motorDrive(int16_t pwmVal, const int m_l, const int m_r) {
  pwmVal = constrain(pwmVal, -1024, 1024);

  if (pwmVal >= 0) {
    analogWrite(m_l, pwmVal);
    digitalWrite(m_r, LOW);
  } else {
    pwmVal = abs(pwmVal);

    digitalWrite(m_l, LOW);
    analogWrite(m_r, pwmVal);
  }
}

/**  stop */
void stopMotor() {
  digitalWrite(M1_l, LOW);
  digitalWrite(M1_r, LOW);
  digitalWrite(M2_l, LOW);
  digitalWrite(M2_r, LOW);
}

/**  前進 */
void goForward(int16_t velocity) {
  motorDrive(velocity, M1_l, M1_r);
  motorDrive(velocity, M2_l, M2_r);
}

/**  後進 */
void goBack(int16_t velocity) {
  motorDrive(-velocity, M1_l, M1_r);
  motorDrive(-velocity, M2_l, M2_r);
}

/**  右旋回 */
void turnRight(int16_t velocity) {
  motorDrive(velocity, M1_l, M1_r);
  motorDrive(-velocity, M2_l, M2_r);
}

/**  右旋回 */
void turnLeft(int16_t velocity) {
  motorDrive(-velocity, M1_l, M1_r);
  motorDrive(velocity, M2_l, M2_r);
}
\end{minted}

\section{おわりに}
ESP-WROOM-02を使って自走式WWEBサーバーを作成することができた.今後の課題として,websocketを用いたリアルタイム性の高い操作を実現したい.また,今回の
