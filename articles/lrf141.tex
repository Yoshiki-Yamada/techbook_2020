\section{はじめに}
このセクションではLinux Kernelに対する知見をカーネルモジュール制作を通して深めることを目的としています。
単純なカーネルモジュールを制作してもあまり意味が無いので、ここではカーネルモジュールとして動作するEchoサーバを作ることを題材にします。

\subsection{カーネルモジュールとは}
カーネルモジュールとは、Linux上で動的につまりは起動中でも追加削除可能なモジュールを指す。
通常のOSではカーネルにモジュールを追加するとカーネルそのものを再構築する必要がでてくるが、Linuxカーネルモジュールはそれを必要とせずにモジュールの追加、利用、削除が可能となっている。
ここで紹介するカーネルモジュールはその特性上、ローダブルカーネルモジュールと呼ばれることがある。
現在ロードされているカーネルモジュールを確かめるには\mintinline{bash}{lsmod}で確認することができる。

\subsection{対応環境}
今回、カーネルモジュールを開発するにあたり使用した環境を以下に示す。
\begin{minted}[frame=lines,framesep=2mm,baselinestretch=1.2,fontsize=\footnotesize,linenos,breaklines]{text}
    linux-headers-4.4.0-116-generic
    gcc (Ubuntu 5.4.0-6ubuntu1~16.04.9) 5.4.0 20160609
    GNU Make 4.1
\end{minted}

使用したMakefileは以下の通り。
\inputminted[frame=lines,framesep=2mm,baselinestretch=1.2,fontsize=\footnotesize,linenos,breaklines]{bash}{\lrfasset/Makefile}

\section{カーネルスレッドを扱う}
ここで作る最速のEchoサーバはソケットを扱い、TCP通信をするという非常に単純な仕様をもつ。
しかし、それだけではつまらないので無駄にスレッドを使っていこうと思う。
純粋にユーザ空間であればそれにあったスレッドを使用すればいいのだが、今回はカーネルモジュールとして動作させるためカーネルスレッドというものを使う。
使用したヘッダーはlinux/kthread.hとlinux/sched.hの2つ。
kthread.hがカーネルスレッドに関する色々を格納していてsched.hがカーネルで扱うプロセスやそれに必要不可欠な構造体を含んでいる。

\begin{minted}[frame=lines,framesep=2mm,baselinestretch=1.2,fontsize=\footnotesize,linenos,breaklines]{c}
#include <linux/kthread.h>
#include <linux/sched.h>

struct task_struct *task;

static int kthread_cb(void *arg){

    printk("[%s] running as kthread\n", task->comm);

    while(!kthread_should_stop()){
        schedule();
    }

    return 0;

}
\end{minted}

グローバル変数として構造体をデータ型に持つ変数を宣言している。これにプロセスとして起動したい処理等を渡す。
上記の関数ではカーネルスレッドとして実際に動作する処理を記述している。それを次のようにモジュール側で使用してみる。

\begin{minted}[frame=lines,framesep=2mm,baselinestretch=1.2,fontsize=\footnotesize,linenos,breaklines]{c}
static int fastecho_init_module(void){

    printk("Fastest Echo Server Start!!");

    //make kernel thread
    task = kthread_create(kthread_cb, NULL, "lrf141:fastecho");
        
    printk("[%s] wake up as kthread\n", task->comm);

    //launch task
    wake_up_process(task);
        
    return 0;
}

static void fastecho_cleanup_module(void){

    printk("Fastest Echo Server is unloaded!");
    printk("[%s] stop kthread\n", task->comm);
    kthread_stop(task);

}
\end{minted}

ここまで書いたものをビルドし、insmodでカーネルに組み込むと以下のような標準出力をdmesgでみることができる。
\begin{minted}[frame=lines,framesep=2mm,baselinestretch=1.2,fontsize=\footnotesize,linenos,breaklines]{bash}
$ make
$ sudo insmod fastecho.ko
$ ps auwx | grep "\[lrf141:fastecho\]"
root     30557 98.3  0.0      0     0 ?        R    16:49   0:05 [lrf141:fastecho]
$ dmesg
[176532.823327] Fastest Echo Server Start!!
[176532.826178] [lrf141:fastecho] wake up as kthread
[176532.827453] [lrf141:fastecho] running as kthread
[176586.304842] Fastest Echo Server is unloaded![lrf141:fastecho] stop kthread
$ sudo rmmod fastecho.ko
\end{minted}

\section{ソケット関連}

前節で、カーネルスレッドを用いてマルチスレッドのような機能を実現できることはわかったのでメインとなるソケットを用いた処理に関連するものを準備する。
bootlin\footnote{\url{https://elixir.bootlin.com/linux/latest/source}}というサイトにカーネルに使用されているヘッダーやソースファイルを閲覧できる機能があるのでこれを用いる。
まずはlinux/net.hというヘッダーにsocket構造体が存在していた。これがカーネルで使用されるソケットらしい。
中身は次のようになっている。
\begin{minted}[frame=lines,framesep=2mm,baselinestretch=1.2,fontsize=\footnotesize,linenos,breaklines]{c}
/**
 *  struct socket - general BSD socket
 *  @state: socket state (%SS_CONNECTED, etc)
 *  @type: socket type (%SOCK_STREAM, etc)
 *  @flags: socket flags (%SOCK_ASYNC_NOSPACE, etc)
 *  @ops: protocol specific socket operations
 *  @file: File back pointer for gc
 *  @sk: internal networking protocol agnostic socket representation
 *  @wq: wait queue for several uses
 */
struct socket {
    socket_state        state;

    kmemcheck_bitfield_begin(type);
    short           type;
    kmemcheck_bitfield_end(type);

    unsigned long         flags;

    struct socket_wq __rcu  *wq;

    struct file     *file;
    struct sock     *sk;
    const struct proto_ops  *ops;
};
\end{minted}

さらにこの文献を漁っていると内部の構造体、\verb|proto_ops|構造体にlistenやopen,bindといったネットワークプログラミングでお馴染みの関数群がある。

これを本来は使っていくべきなのだろうが、素で呼び出すと色々と呼び出すものが増えて厄介になるので、それらのラッパーである\verb|sock_reate|関数や\verb|kernel_bind|関数、\verb|kernel_listen|関数を駆使することとする。

\section{文字列関連}
次に問題になるのが、ソケットで通信周りを扱った後のレスポンスをどう扱うかである。
これには非常に苦戦を強いられた。なぜなら、多くの文献で紹介されているmsghdr構造体が後の仕様変更で大幅に変更されていて従来の方法でiovec構造体を扱うことが出来ないためである。

正確には扱うことができるが、それこそソケット周りのように素で扱うためにはかなりのコードを書く必要がある。
そのため今回は、kvec構造体を使用してリクエストの送受信に\verb|kernel_recvmsg|と\verb|kernel_sendmsg|関数を使用して実現することにした。これを用いることで従来と同様の結果を得ることができるようになった。

実際にカーネルに使用されているヘッダーでは次のように宣言されている。使用方法に関しては、iovecとさほど変わらない。
\begin{minted}[frame=lines,framesep=2mm,baselinestretch=1.2,fontsize=\footnotesize,linenos,breaklines]{c}
struct kvec {
    void *iov_base; /* and that should *never* hold a userland pointer */
    size_t iov_len;
};
\end{minted}
\begin{minted}[frame=lines,framesep=2mm,baselinestretch=1.2,fontsize=\footnotesize,linenos,breaklines]{c}
/*
 *  As we do 4.4BSD message passing we use a 4.4BSD message passing
 *  system, not 4.3. Thus msg_accrights(len) are now missing. They
 *  belong in an obscure libc emulation or the bin.
 */
 
struct msghdr {
    void        *msg_name;  /* ptr to socket address structure */
    int     msg_namelen;    /* size of socket address structure */
    struct iov_iter msg_iter;   /* data */
    void        *msg_control;   /* ancillary data */
    __kernel_size_t msg_controllen; /* ancillary data buffer length */
    unsigned int    msg_flags;  /* flags on received message */
    struct kiocb    *msg_iocb;  /* ptr to iocb for async requests */
};
\end{minted}

\section{Echoサーバの実装}
ここまで解説したものを全て用いて、Echoサーバをカーネルモジュールとして実現したのが以下のコード群である。

最初に記述したソースファイル2つがソケットやスレッド周りの実装となっている。
\inputminted[frame=lines,framesep=2mm,baselinestretch=1.2,fontsize=\footnotesize,linenos,breaklines]{c}{\lrfasset/echo_server.c}
\inputminted[frame=lines,framesep=2mm,baselinestretch=1.2,fontsize=\footnotesize,linenos,breaklines]{c}{\lrfasset/echo_server.h}
次の2つがカーネルモジュールとして動作させるために必要な実装となる。
\inputminted[frame=lines,framesep=2mm,baselinestretch=1.2,fontsize=\footnotesize,linenos,breaklines]{c}{\lrfasset/fastecho.h}
\inputminted[frame=lines,framesep=2mm,baselinestretch=1.2,fontsize=\footnotesize,linenos,breaklines]{c}{\lrfasset/fastecho_module.c}

\section{最後に}
ここまで、色々とカーネルモジュールとしてEchoサーバを作成する方法を書いてきたが、ここに記述したものが正攻法でないと思う。しかし、このようにカーネルモジュールをカーネルに使用されているソースを読み解きながら作成することはLinuxカーネルを理解する第一歩となると思うので、これを読んで興味を持った方はぜひ挑戦してみるといいだろう。


