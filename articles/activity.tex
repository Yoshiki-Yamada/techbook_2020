\section{はじめに}
ここまでお読み頂きありがとうございます.
最後に,ここ最近のLOCAL学生部の活動をご紹介します!

\section{活動報告}
\subsection{SecHack365(1期)}
\begin{description}
\item[参加部員] さわだ
\end{description}

SecHack365は国立研究開発法人情報通信研究機構(NICT)が主催する,若手セキュリティイノベーターを育成する長期ハッカソンです.
その初年度にさわださんが参加しました.\mbox{}\\

\begin{description}
\item[さわださんはSecHack参加者の海外派遣として,SXSWにも参加したそうです!]\mbox{}\\
SecHack365 0xFF SXSW Part.1\mbox{}\\
\url{https://takuzoo3868.hatenablog.com/entry/sechack365_0xff_part1}
\end{description}

\subsection{SecHack365(2期)}
\begin{description}
\item[参加部員] あるねこ・壱・もぐら
\end{description}

同じくSecHackの第2期です.あるねこさん・壱さん・もぐらさんが参加しました.\mbox{}\\

\begin{description}
\item[SecHack沖縄回の帰路,飛行機の中で書いたという,壱さんのブログ.]\mbox{}\\
SecHack365な1年と次の参加者に向けて\mbox{}\\
\url{https://ich1-one.hatenablog.com/entry/2019/04/04/234234}
\end{description}

\begin{description}
\item[あるねこさんの作品は優秀作品に選ばれたそうです!]\mbox{}\\
あるねこさん的SecHack365参加録\mbox{}\\
\url{https://www.aruneko.net/post/sechack_2018/}
\end{description}


\subsection{Web×IoT メイカーズチャレンジ 2018-19 in 札幌}
\begin{description}
\item[参加部員] はいばら
\end{description}

Web×IoT メイカーズチャレンジはIoTシステム開発のスキルアップを目的としたハンズオン\and ハッカソンです.\mbox{}\\

\begin{description}
\item[札幌でのハッカソンで技術賞をいただきました.]\mbox{}\\
Web×IoT メイカーズチャレンジ 2018-19 in 札幌 に行ってきました\mbox{}\\
\url{https://haibara-works.hatenablog.com/entry/2018/11/29/000835}
\end{description}


\subsection{セキュリティ・ミニキャンプ in 北海道}
\begin{description}
\item[参加部員] あるねこ・壱・がっちゃん・はいばら・すとんりばー
\end{description}

セキュリティ・ミニキャンプ(地方大会)は,若年層を対象とした情報セキュリティ人材の育成のための講習会です\mbox{}\\

\begin{description}
\item[セキュリティ・ミニキャンプ in 北海道 2018]\mbox{}\\
\url{https://www.security-camp.or.jp/minicamp/hokkaido2018.html}
\end{description}\mbox{}\\

\begin{quotation}
IoTデバイスのセキュリティやC言語の不適切なコーディングによって産まれる脆弱性,USBデバイスの開発方法を学びました.
また,北海道警察の方から『セキュリティ技術と倫理』についてお話をいただきました.(はいばら)
\end{quotation}

\subsection{HOKKAIDO学生アプリコンテスト2019}
HOKKAIDO学生アプリコンテストは"モバイルアプリケーションの企画力、デザイン力、プログラム力を有する学生を表彰することにより、学生のモバイル分野に対する関心を高め、次代を担う高度なモバイルIT人材の発掘・育成に寄与することを目的"としたコンテストです.\mbox{}\\

\begin{quotation}
アプリコンテストには2回目の参加で、企業賞をいただけて嬉しかったです!次のコンテストに繋げられるようにこれからも頑張りたいです!(ことみん)
\end{quotation}

\subsection{YAPC}
\begin{description}
\item[参加部員] 那由多・ことみん
\end{description}
YAPC(Yet Another Perl Conference)は"Perlを軸としたITに関わる全ての人のためのカンファレンス"です.\mbox{}\\

\begin{description}
\item[那由多のブログ.スクラム開発に興味をもったようです.]\mbox{}\\
いってきましたYAPC::Tokyo 2019\mbox{}\\
\url{https://nayuta-1999.hatenablog.com/entry/2019/01/30/145145}
\end{description}

\begin{description}
\item[ことみんのブログ.前夜祭のLTソンから参加したそうです!]\mbox{}\\
ブログを書くまでがYAPC::Tokyoです。\mbox{}\\
\url{https://kotomi1338.hatenablog.com/entry/2019/01/29/235640}
\end{description}

\subsection{OSC Hokkaido}
オープンソースの文化祭 OSC Hokkaido 2018では学生部でブース展示をしました.また,開催レポートの執筆も担当しました.\mbox{}\\
\begin{description}
\item[昨年より多くの参加者で賑わった OSC2018 Hokkaido]\mbox{}\\
\url{https://www.ospn.jp/press/20180717osc2018-hokkaido-report.html}
\end{description}

\subsection{LOCAL学生部総大会}
LOCAL学生部総大会は,学生部の年一度のオフラインイベントです.2018年度の開催では,さくらインターネットさんの石狩DCで合宿するという,スペシャルな回になりました!\mbox{}\\

\begin{description}
\item[LOCAL学生部のブログ]\mbox{}\\
第10回 LOCAL学生部総大会 開催レポート\mbox{}\\
\url{http://students.local.or.jp/blog/entry/2018/08/23}
\end{description}

\begin{description}
\item[けんつさんのブログ]\mbox{}\\
第10回LOCAL学生部総大会をやってきました\mbox{}\\
\url{http://rabbitfoot141.hatenablog.com/entry/2018/10/06/233550}
\end{description}

\begin{description}
\item[ともかさんのブログ]\mbox{}\\
LOCAL学生部のイベントに初参加してきましたよっ!\mbox{}\\
\url{https://mokomoka.hateblo.jp/entry/localst18}
\end{description}

\begin{description}
\item[はいばらのブログ]\mbox{}\\
行ってきました第10回LOCAL学生部総大会\mbox{}\\
\url{https://haibara-works.hatenablog.com/entry/2018/10/08/113256}
\end{description}

\begin{description}
\item[しかさんのブログ]\mbox{}\\
楽しかった総大会\mbox{}\\
\url{https://sika0115.hatenablog.com/entry/2018/10/07/191213?_ga=2.55916011.1590347690.1538897406-1485988726.1512910189}
\end{description}

\begin{description}
\item[ことみんのブログ]\mbox{}\\
LOCAL学生部の楽しい集まりがありました。\mbox{}\\
\url{https://kotomi1338.hatenablog.com/entry/2018/09/30/225711}
\end{description}

\subsection{技術書典4}
技術書典4にて,学生部の同人誌 第一弾『情報ボーイズの寄稿ノート』を頒布しました.
COMIC ZINにて委託販売も行っておりますので,ご興味のある方は是非お求めください!\url{https://shop.comiczin.jp/products/detail.php?product_id=36441}\mbox{}\\

以下,技術書典4 サークルページから引用.(\url{https://techbookfest.org/event/tbf04/circle/14720002})
\begin{quote}
\begin{description}
\item[概要]\mbox{}\\ 
北海道の学生が集結して作成したオムニバス形式の同人誌.みんな"得意なことは違う"ので,バラエティ豊かな内容になりました.ぜひ手に取ってみてください!

\item[目次]\mbox{}\\ 
\begin{description}
\item[\LaTeX の乱数生成アルゴリズムを調べる]\mbox{}\\ 
\LaTeX 標準の乱数パッケージはFortran由来だった?そんな乱数パッケージの詳細について徹底的に調べてみました.
\item[超入門 仮想化技術]\mbox{}\\ 
仮想化って色々ありすぎてわからない.そんなあなたに仮想化技術の基礎から解説します.
\item[LEDで発電しよう!!!]\mbox{}\\ 
LEDは電力を与えると光るので,逆に光を与えると発電できるのでは.そんな現象を実験で確かめました.
\item[入門Linux Kernel]\mbox{}\\ 
Linux Kernelモジュールの作成方法をサンプルコード付きでわかりやすく解説します.
\item[自作エディタ入門編]\mbox{}\\ 
宗教戦争の果てに,自分だけのエディタを作るお話です.
\item[世界と孤独の説法(エピローグ)]\mbox{}\\ 
PDFで遊べる新感覚のゲーム.紙バージョンでは掲載しきれなかったのでPDF版でお楽しみください.
\item[関数型Python入門]\mbox{}\\ 
PythonでMonadを作りながら,型ヒンティング機能で遊び尽くします.型ヒントを使ってみたいあなたにもオススメ!
\end{description}
\end{description}

\end{quote}
