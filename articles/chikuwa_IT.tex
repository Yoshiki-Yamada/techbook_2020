\section{はじめに}
近年, コンテナ型仮想化の普及もあり仮想化技術という存在を様々な場面でよく聞くようになりました. しかし, コンテナ型仮想化技術以外の他の仮想化技術というのは中々個人で扱うような技術・ソフトウェアではないこともあり, あまり知られていないことが多かったりします. また, コンテナ型仮想化技術も便利で環境構築が魔法のようなすごい存在といった漠然とした解釈の人もきっと多いはずです. そこでこの章では, 各種仮想化技術の仕組みや特徴について触れながらふわっと仮想化技術について紹介します.

\section{仮想化技術とは}
仮想化技術にはいくつか種類がありますが, この章では計算機資源を抽象化してOSなどに見せるプラットフォーム仮想化のことを指します. 仮想化することで複数の計算機資源を単一に見せたり, 単一の計算機資源を複数に見せることができます. そしてプラットフォーム仮想化を支えるための技術としてシステム仮想機械(VM, Virtual Machine)と呼ばれる計算機資源をエミュレートするソフトウェアが存在します. 仮想機械の実装は仮想マシンモニタ(VMM, Virtual Machine Monitor)を始めとしていくつか存在します. 例えば仮想専用サーバ(VPS, Virtual Private Server)のようなサービスでは図2.1にように, 物理サーバ上に仮想化OSによって複数の仮想サーバに分割してユーザに各仮想サーバを提供しています.
\begin{figure}[htbp]
    \centering
    \includegraphics[width=50mm]{./assets/chikuwa_ITasset/vps.png}
    \caption{hogehoge}
    \label{fig:one}
\end{figure}

\section{VMMについて}
仮想化を実現するVMMにはベアメタルハイパーバイザ型(Type1),ホスト型(Type2)の2つに分類されます.

\subsection{ベアメタルハイパーバイザー型(Type1)}
ベアメタルハイパーバイザ型(Type1)は, ハードウェアの上で直接動作します. この方式ではホストOSと呼ばれるような土台になるOSが存在しないため,仮想マシンによる遅延や速度低下を防ぐことができます. そしてベアメタルハイパーバイザでは実装手法でもモノリシックカーネル型とマイクロカーネル型の2つに分類することができるほか, 仮想化のアプローチで完全仮想化と準仮想化に分類することができます.
\subsubsection{モノリシックカーネル型}
主にVMware ESX/ESXiなどで採用されている方式で, モノリシックという英語で「1枚岩」という意味の通り、VMMの中にデバイスドライバが含まれています. VMMがストレージをはじめ,ネットワークや入力デバイスといったハードウェアへのアクセスをすべてを処理します. この方法の利点はVMMとデバイスドライバが密接に連携するため,オーバーヘッドが少なく効率的です. しかしながら, VMMの中にデバイスドライバが存在しているため,ハイパーバイザ層でデバイスドライバを用意する必要があります. そのため, ハードウェアのサポートがマイクロカーネル型と比較して少なく, 使用するハードウェアに制限がかかってしまう場合があります. また, デバイスドライバをVMMに直接組み込むため, バグや脆弱性はVMM全体に広がってしまいます.
\begin{figure}[htbp]
    \centering
    \includegraphics[width=50mm]{./assets/chikuwa_ITasset/monolithic.png}
    \caption{モノリシックカーネル型ベアメタルハイパーバイザ}
    \label{fig:one}
\end{figure}
\begin{figure}[htbp]
    \centering
    \includegraphics[width=50mm]{./assets/chikuwa_ITasset/monolithic.png}
    \caption{モノリシックカーネル型のハードウェアアクセス}
    \label{fig:two}
\end{figure}
\subsubsection{マイクロカーネル型}
主にXenやHyper-Vで採用されている方式で, ハイパーバイザを管理する仮想マシンと管理OSを用意します. この管理OSはLinuxはWindows Serverなど汎用OSを使用します. また, 管理OSはXenではドメイン0,Hyper-Vでは親パーティションと呼ばれています. この方式では, デバイスドライバはVMMではなく, VMM上の仮想マシンとして動作している管理OSのデバイスドライバを使用します. 仮想マシンからハードウェアにアクセスする時はゲストOSから仮想デバイスのインターフェースを経由してVMMから管理OSに渡されます. そして管理OSのデバイスドライバからハードウェアにアクセスします.この方法は, 汎用OSのデバイスドライバを使用することで, モノリシックカーネル型に比べてハードウェアのサポートが多く, ハードウェアの対応が柔軟であるという利点があります. 例えば, Hyper-VならWindows用のデバイスドライバを使用することができます. しかしながら, ハードウェアにアクセスする際にVMMから管理OSを経由するため, モノリシックカーネル型よりも性能が低下してしまいがちであり, 管理用の汎用OSがクラッシュした場合全てのVMがクラッシュしてしまうといった欠点が存在します.
\begin{figure}[htbp]
    \centering
    \includegraphics[width=50mm]{./assets/chikuwa_ITasset/monolithic.png}
    \caption{マイクロカーネル型ベアメタルハイパーバイザ}
    \label{fig:one}
\end{figure}
\begin{figure}[htbp]
    \centering
    \includegraphics[width=50mm]{./assets/chikuwa_ITasset/monolithic.png}
    \caption{マイクロカーネル型のハードウェアアクセス}
    \label{fig:two}
\end{figure}

\subsubsection{完全仮想化}
完全仮想化方式のハイパーバイザでは, ハードウェアの挙動をすべてエミュレートします. そのため, 何も変更も加えていないそのままのホストOSを動かすことができます. 1960年代にIBMが「トラップアンドエミュレート」とよばれる方法で完全仮想化を実装しようとしました. この方法ではゲストOSが特権がない状態(Ring3)で実行させ, 特権(Ring0)が必要な命令を実行しようとすると失敗します. その際にVMMがその失敗をトラップして原因を確認してからその命令をエミュレートすることによってゲストOSの期待する結果を返すことができ, ゲストOSにRing0以外で実行されていることを気づかせないようにすることができます. しかしながら, この手法は古典的ですべてのアーキテクチャに適用できるわけではありませんでした. 特にx86プロセッサの場合, ユーザ権限で実行できるセンシティブ命令と呼ばれる計算機資源の構成などの依存している命令が存在しているため, 実装を難しくさせていました. そこで, 「バイナリトランスレーション」と呼ばれる新しい手法が使われるようになりました.
\subsubsection{準仮想化}
\subsection{ホスト型(Type2)}
