\section{はじめに}
ここまでお読み頂きありがとうございます.最後に2019年度の学生部員の活動報告を掲載します.

\section{活動報告}
\subsection{SecHack365 2019}
\begin{description}
\item[参加部員] はいばら
\end{description}
SecHack365は国立研究開発法人情報通信研究機構(NICT)が主催する,
若手セキュリティイノベーターを育成する長期ハッカソンです.
はいばらは,フィジカルゼミというゼミに所属し,SoC FPGA向けセキュリティ機構の開発をしました.
\mbox{}\\
\begin{description}
\item[SecHack参加についての,はいばらのブログ]\mbox{}\\
(まだブログ書けてない...)\mbox{}\\
\url{https://hugahuga.hogehoge}
\end{description}

\subsection{OSC Hokkaido}
オープンソースの文化祭 OSC Hokkaido 2019でブース展示をしました.
hogehoge
\mbox{}\\
\begin{description}
\item[昨年より多くの参加者で賑わった OSC2018 Hokkaido]\mbox{}\\
\url{https://www.ospn.jp/press/20180717osc2018-hokkaido-report.html}
\end{description}

\subsection{LOCAL学生部総大会}
LOCAL学生部総大会は,学生部の年一度のオフラインイベントです.
hogehoge
\mbox{}\\
\begin{description}
\item[LOCAL学生部のブログ]\mbox{}\\
第11回 LOCAL学生部総大会 開催レポート\mbox{}\\
\url{http://students.local.or.jp/blog/entry/2020/01/23}
\end{description}

\subsection{学生部同人誌}
LOCAL学生部では本誌の他に,過去2冊の同人誌を作成しています.

\subsubsection{2018年度 学生部同人誌『情報ボーイズの寄稿ノート』}
\begin{quote}
\begin{description}
\item[紹介]\mbox{}\\ 
北海道の学生が集結して作成したオムニバス形式の同人誌.
みんな"得意なことは違う"ので,バラエティ豊かな内容になりました!
\item[目次]\mbox{}\\ 
\begin{description}
\item[\LaTeX の乱数生成アルゴリズムを調べる (うっひょい)]\mbox{}\\ 
\LaTeX 標準の乱数パッケージはFortran由来だった?そんな乱数パッケージの詳細について徹底的に調べてみました.
\item[超入門 仮想化技術 (ちくうぇいと)]\mbox{}\\ 
仮想化って色々ありすぎてわからない.そんなあなたに仮想化技術の基礎から解説します.
\item[LEDで発電しよう!!!]\mbox{}\\ 
LEDは電力を与えると光るので,逆に光を与えると発電できるのでは.そんな現象を実験で確かめました.
\item[入門Linux Kernel (あわあわ)]\mbox{}\\ 
Linux Kernelモジュールの作成方法をサンプルコード付きでわかりやすく解説します.
\item[自作エディタ入門編 (さわだ)]\mbox{}\\ 
宗教戦争の果てに,自分だけのエディタを作るお話です.
\item[世界と孤独の説法(エピローグ) (Jumpaku)]\mbox{}\\ 
PDFで遊べる新感覚のゲーム.紙バージョンでは掲載しきれなかったのでPDF版でお楽しみください.
\item[関数型Python入門 (あるねこ)]\mbox{}\\ 
PythonでMonadを作りながら,型ヒンティング機能で遊び尽くします.型ヒントを使ってみたいあなたにもオススメ!
\end{description}
\end{description}
\end{quote}

\subsubsection{2019年度 学生部同人誌『情報ボーイズの寄稿ノート ver.2』}
\begin{quote}
\begin{description}
\item[紹介]\mbox{}\\ 
前回の同人誌は「電子工作から\LaTeX まで」という内容でしたか.
今回の同人誌は「電子工作からUnity まで」になりました.
\item[目次]\mbox{}\\ 
\begin{description}
\item[ESP-WROOM-02で作る 自走式Webサーバー入門 (はいばら)]\mbox{}\\ 
ESP-WROOM-02 は,WiFi とTCP/IP を手軽に扱えるマイコンです.
その一方で,マイコンらしくGPIOやPWMなどの機能も使うことができます.
それなら,自走式Webサーバーを作るのにうってつけですよね?
\item[Docker Remote APIを使用して コンテナを自分で制御する (けんつ)]\mbox{}\\ 
docker-compose,便利ですよね.実はこの同人誌も,
学生部OBのあるねこさん謹製Dockerfile・Docker イメージで\LaTeX の環境を複数の著者で共有しています.
そんなdocker-compose は,その内部でDocker Remote API というDocker コンテナを制御するAPI を使用しています.
このDocker Remote API を使えば,自分でコンテナを制御することもできるのです.
\item[Crack MalwareBytes! WriteUP "MalwareMustDie" (さわだ)]\mbox{}\\ 
SecHack修了生でもある さわださんは,
"Malwarebytes CrackMe 2: try another challenge" というCrack Me のWriteUP を書きました.
マルウェアの解析に興味のある方にはおススメです!
\item[Unityのマスクシェーダーで遊ぼう!! (ゆひ)]\mbox{}\\ 
ゆひくんは,Unity のマスクシェーダーについて書きました.
マスクシェーダーの実装例に加えて,AR アプリへの応用についても紹介されています.
\item[LOCAL学生部 活動報告 (はいばら)]\mbox{}\\ 
当時の学生部員の活動報告.
\end{description}
\end{description}
\end{quote}